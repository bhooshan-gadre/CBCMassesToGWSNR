\documentclass{article}
% Uncomment the following line to allow the usage of graphics (.png, .jpg)
%\usepackage[pdftex]{graphicx}
% Comment the following line to NOT allow the usage of umlauts
\usepackage[utf8]{inputenc}
\usepackage{amsmath}
\usepackage{amssymb}
\usepackage{mathtools}
\usepackage{hyperref}
\usepackage{booktabs}
\everymath{\displaystyle}
%\usepackage{mathrsfs}\\
\DeclareMathAlphabet{\mathpzc}{OT1}{pzc}{m}{it}
\numberwithin{equation}{section}
% Start the document
\begin{document}
\section*{\normalsize{\textit{Abbott et. al.\\}}\small{GW170104: Observation of a 50-Solar-Mass Binary Black Hole Coalescence at Redshift 0.2}}
The second observational run (O2) of aLIGO added 11 more days of coincident observation data and the detection of GW170104 to the observations made in first observational run (O1). Thus, improved BBH merger rate estimates have been obtained. Two simple models of population of BHs have been considered on a representative basis -\\
1. Power law in $m_1$ and uniform in $m_2$,
\begin{center}
	$p(m_1,m_2) \propto m_1^{-\alpha}/(m_1-5M_\odot)$ with $\alpha = 2.35$\\
\end{center}
2. Uniform distribution in logarithms of both of the component masses.\\
\\
In both the cases, $m_1,m_2\geqslant 5M_\odot$ and $M\leqslant100M_\odot$. Taking the observations in O1 as a prior, the updated rate estimates are calculated as $R = 103^{+110}_{-63}$ Gpc$^{-3}$yr$^{-1}$ and $R = 32^{+33}_{-20}$ Gpc$^{-3}$yr$^{-1}$ for the two models respectively. Also, the new range of event rate which caps the two distributions is calculated to be $12 - 213$ Gpc$^{-3}$yr$^{-1}$ and is consistent with the previous range $9 - 240$ Gpc$^{-3}$yr$^{-1}$, calculated from O1.\\
\\
With the addition of GW170104 to the previous detections GW150914, LVT151012 and GW151226, the value of $\alpha$ in power law distribution (as discussed above) is updated from $2.5^{+1.5}_{-1.6}$ to $2.3^{+1.3}_{-1.4}$ (Section IV of Supplement Material\cite{1})


\section*{\normalsize{\textit{Abadie et. al.\\}}\small{Predictions for the Rates of Compact Binary Coalescences Observable Ground-based Gravitational-wave Detectors}}
Astrophysical predictions of CBC merger rates depend on a number of assumptions and unknown model parameters and are still ambiguious. This paper published in 2010, summarizes all the predictions made about CBC merger and detection rates from 2000 to October 1 2009. The most confident estimates are those of binary neutron stars because of the extrapolations of observed BNS systems in our Galaxy.\\
\\
CBC rates are assumed to be proportional to the stellar birth rates in spiral galaxies in the vicinity, which can be determined by their blue light luminosity L$_{B, \odot}$. Thus, the rates are expressed as coalescence rate per unit L$_{10}$ ($10^{10}$ times solar blue light luminosity, L$_{B, \odot})$. This method, however, does not track the star formation rate in the past very accurately as it does not consider the contributions of older populations from elliptical galaxies. Thus predictions made here are mainly dependent on the contribution from spiral galaxies and thus the rates are calculated in the units MWEG$^{-1}$ or L$_{10}^{-1}$. In future, when contribution from elliptical galaxies is properly included in the literature, overall rates will, more naturally, be qouted in Mpc$^{-3}$. Thus, rates dependent on spiral galaxies can be qouted in units Mpc$^{-3}$ Myr$^{-1}$ as follows,

\begin{table}[h!]
	\centering
	\caption{Compact binary coalescence rates Mpc$^{-3}$ Myr$^{-1}$}
	\label{tab:table1}
	\begin{tabular}{ccccc}
		\toprule
		Source & $R_{low}$ & $R_{re}$ & $R_{high}$ & $R_{max}$\\
		\midrule
		NS-NS (Mpc$^{-3}$ Myr$^{-1}$) & 0.01 & 1 & 10 & 50\\
		NS-BH (Mpc$^{-3}$ Myr$^{-1}$) & 6 $\times$ 10$^{-4}$ & 0.03 & 1 & \\
		BH-BH (Mpc$^{-3}$ Myr$^{-1}$) & 1 $\times$ 10$^{-4}$ & 0.005 & 0.3 & \\
		\bottomrule
	\end{tabular}
\end{table}

Taking the minimum masse of Neutron star as 1.4 M$_\odot$ and that of Black Hole as 10 M$_\odot$, the horizon distances are calculated for Initial / Advanced LIGO as,

\begin{table}[h!]
	\centering
	\caption{Horizon distances for binary sources in Initial / Adv LIGO}
	\label{tab:table1}
	\begin{tabular}{cc}
		\toprule
		Type of binary source & Horizon distance\\
		\midrule
		NS-NS & 33 Mpc / 445 Mpc\\
		NS-BH & 70 Mpc / 927 Mpc\\
		BH-BH & 161 Mpc / 2187 Mpc\\
		\bottomrule
	\end{tabular}
\end{table}

The volumes, thus found out, gives us the detection rates as follows,

\begin{table}[h!]
	\centering
	\caption{Estimated detection rates}
	\label{tab:table1}
	\begin{tabular}{cccccc}
		\toprule
		 & Source & $\dot{N}_{low}$ & $\dot{N}_{re}$ & $\dot{N}_{high}$ & $\dot{N}_{max}$\\
		 & & yr$^{-1}$ & yr$^{-1}$ & yr$^{-1}$ & yr$^{-1}$\\
		
		\midrule
		 & NS-NS & $2 \times 10^{-4}$ & 0.02 & 0.2 & 0.6\\
		Initial LIGO & NS-BH & $7 \times 10^{-5}$ & 0.004 & 0.1 & \\
		 & BH-BH & $2 \times 10^{-4}$ & 0.007 & 0.5 & \\
		
		\midrule
		& NS-NS & 0.4 & 40 & 400 & 1000\\
		Adv LIGO & NS-BH & 0.2 & 10 & 300 & \\
		& BH-BH & 0.4 & 20 & 1000 & \\
		\bottomrule
	\end{tabular}
\end{table}

\subsection*{\normalsize{Derivation of BBH merger rates}}
There are two possible pictures of a BH-BH merger. One is the isolated binary-evolution scenario, which is expected to be the dominant case for BNS and NS-BH mergers. The second and more significant one for BBH systems, considering their high mass, is the dynamical-formation scenario. Dynamical interactions in dense stellar zones, like globular and nuclear star clusters, contribute significantly to the formation and/or hardening of BBH systems. But, due to large uncertainties in the estimates of dynamical-formation scenario and difficulties in determining ranges, considering limited number of models, the second scenario is not included in the prediction tables mentioned above.

\subsubsection*{1. Isolated binary-evolution scenario}
As there are no direct observations of BH-BH binary coalescences yet (till the time of writing the paper), BH-BH rate estimates can only rely upon population synthesis models.The delay between binary star formation and their eventual merger being quite large, current elliptical galaxies with low blue light luminosity can, as well, contribute to BBH merger rate significanly.\\
Though, it does not account for the delay between star birth and mergers properly, the results from \cite{14} have been considered here. Values of $R_{low}$, $R_{re}$ and $R_{high}$ are found out by visual inspection of Figure 15 of \cite{14} as,\\
$R_{low} = 0.01$, $R_{re} = 0.4$ and $R_{high} = 30$.\\
Apart from \cite{14}, several other results have been studied. They are tabulated below-

\begin{table}[h!]
	\centering
	\caption{Predictions of BH-BH inspiral rates}
	\label{tab:table1}
	\begin{tabular}{ccccc}
		\toprule
		Rate model & $\dot{R}_{low}$ & $\dot{R}_{re}$ & $\dot{R}_{high}$ & $\dot{R}_{max}$\\
		
		\midrule
		O’Shaughnessy et al. pop. synth. \cite{14} (MWEG$^{-1}$ Myr$^{-1}$) 			& 0.01 & 0.4 & 30 &\\
		Voss \& Tauris pop. synth. \cite{34} (MWEG$^{-1}$ Myr$^{-1}$)					& 1.3 & 9.7 & 76 &\\
		Belczynski et al. pop. synth.: model A of \cite{35} (MWEG$^{-1}$ Myr$^{-1}$) 	& & 0.02 & & \\
		Belczynski et al. pop. synth.: model B of \cite{35} (MWEG$^{-1}$ Myr$^{-1}$) 	& & 0.02 & & \\
		Belczynski et al. pop. synth.: model C of \cite{35} (MWEG$^{-1}$ Myr$^{-1}$) 	& & 0.02 & & \\
		Nelemans pop. synth. \cite{36} (MWEG$^{-1}$ Myr$^{-1}$) 						& 0.1 & 5 & 250 & \\
		“Double-core” scenario: Dewi et al. \cite{37} (MWEG$^{-1}$ Myr$^{-1}$)			& 0.19 & 19.87 & &\\
		Globular cluster dynamics \cite{55} (Mpc$^{-3}$ Myr$^{-1}$)						& 10$^{-4}$ & 0.05 & & 1\\
		Globular cluster dynamics and pop. synth. \cite{42} (GC$^{-1}$ Gyr$^{-1}$) 		& & 2.5 & & \\
		Nuclear cluster w/ MBH \cite{56} (NC$^{-1}$ Myr$^{-1}$)							& 2$\times10^{-4}$ & 1.3$\times10^{-3}$ & 0.015 &\\
		Nuclear cluster w/out MBH \cite{57} (NC$^{-1}$ Myr$^{-1}$)						& & 0.3 & &\\
		\bottomrule
	\end{tabular}
\end{table}



\newpage
\bibliographystyle{plain}
\bibliography{info}
\end{document}